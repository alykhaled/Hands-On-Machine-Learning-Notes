\documentclass{article}

\usepackage{amsmath} 
\usepackage[a4paper, total={6in, 8in}]{geometry}
\title{Chapter 8: Dimensionality Reduction} 
\author{Aly Khaled} 
\date{\today}
 

\begin{document} 
    \maketitle
    \section{The Curse of Dimensionality}
    \begin{itemize}
    	\item The more dimensions the training set has, the greater the risk of \textbf{overfitting} it.
    	\item One solution to the curse of dimensionality could be to \textbf{increase} the size of the training set to reach a sufficient density of training instances.
    	\item Unfortunately, the number of training instances required to reach a given density grows \textbf{exponentially} with the number of dimensions.
    \end{itemize}
    \section{Approaches for Dimensionality Reduction}
    \begin{itemize}
    	\item There are two main approaches to reducing dimensionality \textbf{Projection} and \textbf{Manifold}.
	\end{itemize}
	\subsection{Projection}
	\begin{itemize}
		\item 
	\end{itemize}	        
	\subsection{Manifold Learning}
	\begin{itemize}
		\item 
	\end{itemize}
	\section{Dimensionality reduction algorithms:}	
	\begin{itemize}
		\item There are many dimensionality reduction algorithms like: PCA, Kernel PCA, LLE
	\end{itemize}
						
    \section*{Papers to read later}
    \begin{itemize}
    	\item Karl Pearson, "On Lines and Planes of Closet Fit to System of Points in Space"
    \end{itemize}

\end{document}